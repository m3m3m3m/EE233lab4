\paragraph{Audio Equalizer Filters}
We will begin the lab by building three equalizer filters. The center frequencies of the three filters should be around 250$ \si{\hertz} $, 1$ \si{\kilo\hertz} $ and 4$ \si{\kilo\hertz} $. To meet the requirement, we have already calculated and simulated for the proper capacitors in prelab exercises. To justify our choice, we will need to record the amplitude of the input and output signals and the corresponding frequencies to plot the figure of gain in terms of frequency. 
\paragraph{Audio Mixer}
After the equalizer filters are built, the last part necessary for an audio mixer is finished. We can build the audio mixer now. According to the instruction of \textbf{Overview of Audio Mixer.pdf}, we will need to build three filters, a buffer and a summing amplifier to form an equalizer, along with a preamplifier and another summing amplifier, which means that we will need to use 7 op-amps in this circuit. We will need to build a very compact circuit while considering not putting the danger of shorting circuits in the final circuit. After we build the circuit, we can connect the input sources to different cellphones and the output to the speaker borrowed from the EE store.